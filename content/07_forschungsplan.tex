\chapter{Forschungsplan}

Um die zwei verschiedenen Techniken zu vergleichen soll ein Testnetzwerk aufgebaut werden, welches genutzt werden soll um diese hinsichtlich ihrer verschiedenen Eigenschaften zu testen.
Das Netzwerk soll zwei Nachbargebäude der Hochschule verbinden um als Prototyp für größere Netzwerke dienen zu können.
Ein Großteils des Forschungsaufwandes wird während des Aufbaus und der Inbetriebnahme, vor allem hinsichtlich der Messbarkeit der zu ermittelnden Kenngrößen, betrieben.

Während dem Aufbau und Betrieb werden sowohl der Aufwand als auch alle Kosten dokumentiert und verglichen.

In der einjährigen Testphase werden Kenngrößen wie Datendurchsatz, Zuverlässigkeit und Energiebedarf ermittelt.
Zusätzlich soll die Sicherheit mit mehreren Experimenten untersucht werden.
Hierzu sollen bereits gefundene Sicherheitslücken ausgenutzt und hinsichtlich ihrer Wirksamkeit geprüft werden.

Der Aufwand für die Attacken und der Erfolg dieser wird in Relation gesetzt um eine Wahrscheinlichkeit für eine reale Durchführung eines solchen Angriffes abschätzen zu können.

Die Wahrscheinlichkeit eines Angriffes wird anschließend mit den anderen Kenngrößen ins Verhältnis gesetzt um die Verhältnismäßigkeit von Nutzen und Aufwand darzustellen.

