\chapter{Hypothesen}

Die zwei unterschiedlichen Übertragungsmedien bieten unterschiedliche Vor- und Nachteile.
Glasfaser bietet die Möglichkeit das Medium geschützt vor unbefugten Zugriff zu verlegen.
Das bedeutet, dass ein Glasfaser Kabel speziell für diese Verbindung verlegt wird.
Das bedeutet, der Aufwand und die Kosten für die Inbetriebnahme des Netzwerkes steigen.
Alternativ kann unter Umständen auch auf vorhandene Infrastruktur zurückgegriffen werden, diese muss aber direkt zwischen den zwei Teilnehmern verlegt werden und darf nicht von den herkömmlichen Kommunikationsmitteln genutzt werden.
Dass ein physikalischer Zugriff durch die Installation weitestgehend ausgeschlossen wird macht es aufwendiger Angriffe auf die Verbindung auszuführen.
Dagegen braucht eine Übertragung über Luft nur zwei Antennen, eine Auf der Sende, die andere auf der Empfänger Seite.
Die Nutzung von dem shared medium Luft macht es einem Angreifer aber leicht, die Übertragung zu stören, manipulieren oder abzuhören.

Daher wollen wir untersuchen, ob ein Quantennetzwerk über Luft genauso sicher ist, wie ein Quantennetzwerk über Lichtwellenleiter.
Und ob der Gewinn an Sicherheit dem Aufwand und den Kosten der Installation entspricht.

Des weiteren benötigt ein Netzwerk welches mittels Funktechnologie arbeitet mehr Energie beim betreiben des Netzwerkes und die Antennen müssen regelmäßig gewartet werden.
