\chapter{Kernfragestellung}

In Zeiten der immer größer werdenden Rechenkapazitäten von Computern 
werden gängige Verschlüsselungsmechanismen immer unsicherer. In diesem 
Kontext scheinen auf Quantenverschränkung basierende Verschlüsengsverfahren
abhilfe zu schaffen. Diese Verfahren benötigen für ihren Einsatz neuartige 
Netzwerkstrukturen zur Übertragung Quantenverschränkter Signale. 
Sogennante “Quantennetzwerke” versperechen ein außergewöhnliche 
Sicherheitsniveau welches für herkömmliche Netzwerke unerreichbar ist.
Die Verschlüsselungalgorithmen werden in den Medien oft als “unknackbar” 
angepriesen. In der Praxis wurde diese Behauptung jedoch bereits mehrfach widerlegt.

In der Praxis wurde diese Behauptung jedoch bereits mehrfach widerlegt.
In unserem Forschungsprojekt soll erörtert werden wie ein solches Netz funktioniert und beispielhaft für die Hochschule München implementiert werden könnte.
Hieraus ergeben sich mehrere technische Möglichkeiten welche hinsichtlich ihrer funktionsweise und ihren Sicherheitsmerkmalen erprobt und getestet werden müssen.

Bei der Implementierung eines Quantennetzwerkes sind neben der Sicherheit des Netzes auch weitere Aspekte zu beachten.
So wollen wir neben der Sicherheit auch die Nachhaltigkeit der verschiedenen Techniken vergleichen.
Im speziellen sind hier die Aufwände bei der Inbetriebnahme, die benötigte Energie während dem Betrieb und die Zuverlässigkeit zu nennen.
