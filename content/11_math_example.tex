\chapter{Math}

\begin{eqnarray*}
	f(x) = \sum_{i=0}^{n} \frac{a_i}{1+x} \\
	\textstyle f(x) = \textstyle \sum_{i=0}^{n} \frac{a_i}{1+x} \\
	\scriptstyle f(x) = \scriptstyle \sum_{i=0}^{n} \frac{a_i}{1+x} \\
	\scriptscriptstyle f(x) = \scriptscriptstyle \sum_{i=0}^{n} \frac{a_i}{1+x}
\end{eqnarray*}

\section{Inline math}
In this line is a beautiful formular $e^{i\pi} + 1 = 0$ \par
This is also inline but special \[e^{i\pi} + 1 = 0\] Do you see it?

\section{Formulars}
For not only mention a formular, you should use that format:

\begin{equation}
E=mc^2
\end{equation}

\section{Multiline}

For long equations
\begin{multline}
p(x) = 3x^6 + 14x^5y \\+ 590x^4y^2 + 19x^3y^3
- 12x^2y^4 - 12xy^5 + 2y^6 \\- a^3b^3 + a^3 - b^6
\end{multline}


Or multipart
\begin{equation}
\begin{split}
A & = \frac{\pi r^2}{2} \\
& = \frac{1}{2} \pi r^2
\end{split}
\end{equation}

Or on another way:


\begin{align}
x&=y           &  w &=z              &  a&=b+c\\
2x&=-y         &  3w&=\frac{1}{2}z   &  a&=b\\
-4 + 5x&=2+y   &  w+2&=-1+w          &  ab&=cb
\end{align}

Without alignment:

\begin{gather} 
2x - 5y =  8 \\ 
3x^2 + 9y =  3a + c
\end{gather}

\section{spacing}
\begin{align*}
f(x) =& x^2\! +3x\! +2 \\
f(x) =& x^2+3x+2 \\
f(x) =& x^2\, +3x\, +2 \\
f(x) =& x^2\: +3x\: +2 \\
f(x) =& x^2\; +3x\; +2 \\
f(x) =& x^2\ +3x\ +2 \\
f(x) =& x^2\quad +3x\quad +2 \\
f(x) =& x^2\qquad +3x\qquad +2
\end{align*}

\section{Brackets}

\ldots need to be equalized even with invisible brackets

\begin{align}
y  = 1 + & \left(  \frac{1}{x} + \frac{1}{x^2} + \frac{1}{x^3} + \ldots \right. \\
& \quad \left. + \frac{1}{x^{n-1}} + \frac{1}{x^n} \right)	
\end{align}

\section{Units}
\SI{24,38176541e4}{m.Pa} \\
\num{24,38176541e4} \\
\si{\kilo\gram\metre\per\square\second} \\
\si{\gram\per\cubic\centi\metre}        \\
\si{\square\volt\cubic\lumen\per\farad} \\
\si{\metre\squared\per\gray\cubic\lux}  \\
\si{\henry\second} \\
\SI[mode=text]{1.23}{J.mol^{-1}.K^{-1}}          \\
\SI{.23e7}{\candela}                              \\
\SI[per-mode=symbol]{1.99}[\$]{\per\kilogram}    \\
\SI[per-mode=fraction]{1,345}{\coulomb\per\mole} \\
\si[unit-color=blue]{\highlight{red}\kilogram\metre\per\second}
$p=\SI{24,381}{MPa}$

\section{list/Range}
\numlist{10;30;50;70} \\
\numrange{10}{30}\\
\SIlist{10;30;45}{\metre}\\
\SIrange{10}{30}{\metre} \\
\num[negative-color = red]{-15673}

\section{Angle}
\ang{10}    \\
\ang{12.3}  \\
\ang{4,5}   \\
\ang{1;2;3} \\
\ang{;;1}   \\
\ang{+10;;} \\
\ang{-0;1;}









