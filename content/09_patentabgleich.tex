\chapter{Patentabgleich}

In diesem Kapitel soll kurz auf zwei mit den Techniken verbundene Patente eingegangen werden und wie die Patentrecherche bewerkstelligt wurde.
Es wurden zu jeder der Technologien, welche wir Vergleichen, ein Patent ausgewählt welches repräsentativ für diese Technologie steht.

\section{Patent zur Übertragung über Optische Medien}

Das Patent EP000001987616A2 schützt eine Technik um sowohl klassische Signale als auch Quanten Signale über das selbe Optische Netzwerk zu verschicken und wurde 2007 angemeldet.

Besonders interessant ist dieses Patent im Zusammenhang mit unserer Arbeit, da eine solche Technik bedeutet, dass die Schlüssel über die selbe Datenverbindung miteinander ausgetauscht werden können.
Dies bedeutet auch, dass für Netzwerke ggf.~keine neuen Leitungen verlegt werden müssen, sondern nur die Endpunkte ersetzt werden.
Dieses Patent bedeutet, dass der Vorteil der Übertragung der Photonen über Luft, dass keine neuen Kabel verlegt werden müssen entfällt, wenn bereits eine direkte \ac{LWL} Verbindung zwischen den Endpunkten existieren.
Dieses Patent behandelt eine Technik, die schon eine Weiterentwicklung des einfachen Quantenschlüsselaustauschs darstellt, weshalb die Grundlegende Technik als Stand der Technik beschrieben werden kann.

Auf dieses Patent sind wir in DEPATIS gestoßen.
Dabei haben wir in mittels der Einsteigerrecherche nach \glqq Quanten Netzwerk\grqq~gesucht.
Diese Suche ergab vier Treffer, wovon drei über die Verteilung von Schlüsseln in einem Quanten Netzwerk gehen.

\section{Patent zur Übertragung über die Luft}

Mit dem, noch relativ jungen Patent aus 20017, CN107508674A wird eine mögliche Technik geschützt, die es ermöglicht Quantenschlüssel über die Luft zu verteilen.
Die Vielzahl der Patente in dem Bereich und dass es Patente gibt, die verschiedene Techniken beschreiben um einen Quantenschlüssel drahtlos auszutauschen zeigt, dass eine solche Methode inzwischen den Stand der Technik erreicht hat.

Dieses Patent wurde im Expacenet gefunden.
Dafür wurden die Begriffe \glqq Quantum wireless network \grqq~in das Suchfeld eingegeben.
Nachdem diese Suche auch sehr viele Ergebnisse in anderen Bereichen hatte, wurde das Wort \glqq key \grqq~hinzugefügt.
Dadurch wird die Suche auf Methoden eingegrenzt die mit einem Schlüssel zu tun haben, welches in dem Kontext der ersten Wörter Ergebnisse liefert die mit einem Schlüsselaustausch in einem Quantennetzwerk zu tun haben.
