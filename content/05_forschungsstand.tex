\chapter{Aktueller Forschungsstand}

Aktuell werden zwei verschiedene Ansätze verfolgt.
Entweder werden Photonen über Lichtwellenleiter an den Empfänger gesendet oder die Photonen werden durch die Luft gesendet.
Die zwei Methoden haben unterschiedliche Vor- und Nachteile, so ist die Übertragung durch einen Lichtwellen Leiter auf weniger 100km beschränkt und die Übertragung durch die Luft benötigt eine Strecke ohne feste, störende Objekte.

\section{Air}
\section{Lichtwellenleiter}

Die Standardmethode für verschlüsselte Kommunikation in Quantennetzwerken ist die \ac{QKD} über Fiberglas.
Hier wird die grundlegende Materie des Lichts, Photonen, übertragen um Informationen weiter zu geben.
Diese Übertragung kann nicht ohne Rauschen stattfinden, sodass ein Übertragung bis maximal 100 km möglich ist\cite{Shen2018}.

Des weiteren findet die Übertragung immer von einem Teilnehmer zu einem zweiten statt und ein Multicast, welcher eine Information von einem Sender an mehrere Empfänger schickt, ist in Quantennetzwerken mit \ac{QKD} nicht etabliert.
Stattdessen werden die Informationen individuell an jeden einzelnen Empfänger gesendet oder Daten werden von einem Verteiler an die verschiedenen Empfänger verteilt.
Dabei muss ein Schlüsselaustausch zwischen dem ersten Teilnehmer und dem Verteiler stattfinden und ein zweiter Schlüsselaustausch zwischen dem Verteiler und dem zweiten Teilnehmer.
Somit ist die Übertragung auf dem Verteiler unverschlüsselt und deshalb muss dem Verteiler vertraut werden\cite{Qui2018}.

Dieses Jahr wurde ein Netzwerk zwischen acht Teilnehmern realisiert, welches mit multiplexern und demultiplexern arbeitet.
Das hat den Vorteil, dass keine aktives verteilen der einzelnen Daten geschehen muss\cite{Siddarth2020}.

Die Übertragung in einem Lichtwellenleiter kann dabei entweder im single-mode als auch im multi-mode stattfinden.
Beim single-mode wird ein einzelner Photonenstrahl vom Sender in den Lichtwellenleiter gegeben, was ein dünneres Kabel erlaubt.
Allerdings macht die mulit-mode Methode eine höhere Präzession möglich\cite{VanMeter2014}.
Beide Möglichkeiten werden aktuell schon in klassischen Netzwerken genutzt und sind weit verbreitet.


