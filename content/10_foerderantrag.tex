\chapter{Förderantrag}\label{foerderantrag}

In diesem Kapitel soll erklärt werden warum wir uns mit unserem Förerantrag an den ausgewählten Förderer richten.

Die genutzten Technologien sind bereits über ihre Konzeptionelle Phase hinaus.
Des weiteren sind die Funktionsweisen auch Experimentel schon bestätigt worden und erste Demonstrationen wurden durchgeführt.
Aber in der Industriellen Nutzung sind die Technologien nicht angekommen, wesshalb wir uns im sogenannten \ac{TRL} 6 sehen.

Damit entfallen die meisten Stiftungen, da diese meist Grundlagenforschung finanzieren.
Aber die staatlichen Förderer sind dafür bekannt Projekte in diesem \ac{TRL} zu fördern.

Auf Länderebene wäre hier die Bayerische Forschungsstiftung zu nennen, welche bei der Informations- und Kommunikationstechnologie noch kein Projekt im Bereich der Quantenkommunikation hat~\cite{bayerischeForschungsstiftung}.
Allerdings fördert sie Projekte welche dazu dienen Forschungsergebnisse in Produkte umzusetzen und nur in Außnahmefällen die Durchführung von Studien oder experimenteller Entwicklung~\cite{bayerischeForschungsstiftung}.

Auf Bundesebene hat das Rahmenprogramm \glqq Selbstbestimmt und sicher in der digitalen Welt 2015-2020\grqq~den Forschungsschwerpunkt \glqq Hightech-Technologien für IT-Sicherheit \grqq~mit dem Bereich der Quantenkommunikation~\cite{BundRahmenprogramm}.
Der Fördergeber ist hier das \ac{BMBF}.
Im Fokus der Förderung steht bei diesem Programm Projekte die IT-Sicherheit sowohl bei klassischen Netzwerken und Protokollen, wie auch bei der Quantenkommunikation.
Gefördert werden alle Projekte die innovative Bausteine und Gesamtlösungen erforschen.

Außerdem bietet die Europäische Kommission im \glqq Horizon 2020\grqq~Programm ein Förderung im Bereich Forschung und Innovation an.
Eines der Schwerpunkte dieses Programms beinhaltet den Bereich der Informations- und Kommunikationstechnologie in dem wir uns mit unserem Vorhaben befinden.

Wir haben uns hier für die Ausschreibung des Bundes entschieden, da nicht nur der Themenbereich sehr gut passt sondern auch die Chance auf eine 100\% Förderung der Ausgaben besteht, sondern auch eine Projektpauschale von 20\% möglich ist.


